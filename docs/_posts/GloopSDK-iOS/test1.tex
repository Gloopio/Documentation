\documentclass[]{article}
\usepackage{lmodern}
\usepackage{amssymb,amsmath}
\usepackage{ifxetex,ifluatex}
\usepackage{fixltx2e} % provides \textsubscript
\ifnum 0\ifxetex 1\fi\ifluatex 1\fi=0 % if pdftex
  \usepackage[T1]{fontenc}
  \usepackage[utf8]{inputenc}
\else % if luatex or xelatex
  \ifxetex
    \usepackage{mathspec}
  \else
    \usepackage{fontspec}
  \fi
  \defaultfontfeatures{Ligatures=TeX,Scale=MatchLowercase}
\fi
% use upquote if available, for straight quotes in verbatim environments
\IfFileExists{upquote.sty}{\usepackage{upquote}}{}
% use microtype if available
\IfFileExists{microtype.sty}{%
\usepackage{microtype}
\UseMicrotypeSet[protrusion]{basicmath} % disable protrusion for tt fonts
}{}
\usepackage{hyperref}
\hypersetup{unicode=true,
            pdftitle={GloopObjects},
            pdfborder={0 0 0},
            breaklinks=true}
\urlstyle{same}  % don't use monospace font for urls
\usepackage{longtable,booktabs}
\IfFileExists{parskip.sty}{%
\usepackage{parskip}
}{% else
\setlength{\parindent}{0pt}
\setlength{\parskip}{6pt plus 2pt minus 1pt}
}
\setlength{\emergencystretch}{3em}  % prevent overfull lines
\providecommand{\tightlist}{%
  \setlength{\itemsep}{0pt}\setlength{\parskip}{0pt}}
\setcounter{secnumdepth}{0}
% Redefines (sub)paragraphs to behave more like sections
\ifx\paragraph\undefined\else
\let\oldparagraph\paragraph
\renewcommand{\paragraph}[1]{\oldparagraph{#1}\mbox{}}
\fi
\ifx\subparagraph\undefined\else
\let\oldsubparagraph\subparagraph
\renewcommand{\subparagraph}[1]{\oldsubparagraph{#1}\mbox{}}
\fi

\title{GloopObjects}
\date{}

\begin{document}
\maketitle

\section{GloopObject}\label{gloopobject}

A GloopObject is any Object that that subclasses (inherits) from the
\texttt{GLPObject} class OR adheres to (implements) the
\texttt{\textless{}GLPObject\textgreater{}} protocol.

\subsection{Storing data}\label{storing-data}

GloopObjects are saved automatically when they are created.

We try to keep an accurate representation of GloopObject in the database
- both locally and remotely.

\subsubsection{Example}\label{example}

\subparagraph{Objective-C Schema:}\label{objective-c-schema}

\begin{verbatim}
@interface Person : NSObject<GLPObject> {
    int shoe_size;
}

@property (nonatomic) NSString *name;
@property (nonatomic) NSNumber *age;

@end
\end{verbatim}

\subparagraph{Objective-C runtime:}\label{objective-c-runtime}

\begin{verbatim}
Person *aPerson = [Person new];
aPerson.show_size = 8;
aPerson.name = @"John";
aPerson.age = @32;
\end{verbatim}

\subparagraph{SQL representation:}\label{sql-representation}

\texttt{Person}

\begin{longtable}[]{@{}llll@{}}
\toprule
objectId & shoe\_size & name & age\tabularnewline
\midrule
\endhead
f9bfca3c-3fdf-4586-9a43-7de6034cf804 & 8 & John & 32\tabularnewline
\bottomrule
\end{longtable}

\subsubsection{Supported Datatypes}\label{supported-datatypes}

Each property that belongs to a GloopObjects can be one of the following
types: \textbf{bool, int, float, double, NSString, NSDate, NSData, and
NSNumber} Other datatypes are be supported using CustomSerializers, see
here.

\subsubsection{Relationships}\label{relationships}

GloopObjects can maintain relationships with other GloopObjects. When a
GloopObject has a property that is also a GloopObject, it is called a
parent-child relationship. If the parent object is saved, the child
object will automatically be saved. A single GloopObject can be both a
parent and a child at the same time.

In a parent-child relationship, the child is saved independantly and a
reference to the child is stored inside the parent. Although internally
the property on the parent is just a reference to the child, it is
indistinguishable from the real child.

\subsection{Retrieving data}\label{retrieving-data}

Retrieving data is done with queries

\end{document}
